\input tex2page %-*-tex-*-
\htmlstylesheet{../tex2page/tex2page.css}

\let\q\scmp
\scmkeyword{my-let my-or either}

\subject{Scheme Macros for Common Lisp}

\urlh{mbe-lsp.tar.gz}{[Download mbe.lsp]}

\smallskip

\urlh{http://www.cs.rice.edu/~dorai}{Dorai Sitaram}

\bigskip

\p{mbe.lsp} defines for Common Lisp the macro definers \q{define-syntax},
\q{let-syntax}, and \q{letrec-syntax}, as described in the
Scheme report R5RS \cite{r5rs}.

These macro definers, also called {\it macro by example
(MBE)\/}, use simple patterns, including ellipsis, to
specify how a macro should be expanded.  They were
propounded by Eugene Kohlbecker \cite{kohlbecker86,kohlbecker-thesis}
in the mid-1980s and became part of
the Scheme standard in 1991.

MBE contrast dramatically with Common Lisp's \q{defmacro}
and
\q{macrolet}, which require the user to program expansions
using extensive list destructuring and restructuring, often
involving several nestings of backquotes, quotes, unquotes
and spliced unquotes.

For example, a \q{let} macro could be defined using MBE
as follows (we will use \q{my-let} in order not to clash
with the standard \q{let}):

\q{
(define-syntax my-let
  (syntax-rules ()
    ((my-let ((x v) ***) e ***)
     ((lambda (x ***) e ***) v ***))))
}

The \q{***} indicates an {\it ellipsis}.  Thus, the input pattern \q{((x v)
***)} matches a list, each of whose elements match \q{(x
v)}.  When expanding, the output pattern \q{(x ***)}
represents a list containing all the \q{x}'s in the same
order that they matched the \q{x}'s in the input
pattern.\numfootnote{The ellipsis is represented by `\q{...}'
in Scheme.  This Common Lisp implementation uses
`\q{***}' instead, since `\q{...}' is disallowed by Common
Lisp's identifier conventions.}

Using \q{defmacro}, the definition would be:

\q{
(defmacro my-let (xvxv &rest ee)
  `((lambda ,(mapcar #'car xvxv)
      ,@ee)
    ,@(mapcar #'cadr xvxv)))
}

\section{\tt define-syntax}

Global MBE macros are defined using the form
\q{define-syntax}.  Its first subexpression is the name of the
macro to be defined, and its second subexpression is the
\q{syntax-rules} governing that macro's expansion.

The first subexpression of \q{syntax-rules} is a list of
auxiliary {\it keywords\/} pertaining to the macro.  The
remaining subexpressions of \q{syntax-rules} are the macro
expansion {\it clauses}.  Each clause consists of an {\it
in-pattern\/} and an {\it out-pattern}.  If the macro call
matches an in-pattern, it expands to the corresponding
out-pattern.  The clauses are tried in sequence: a macro
call expands based on the first in-pattern that it matches.

Here is an example of a disjunction form \q{my-or} (same as
CL \q{or}):

\q{
(define-syntax my-or
  (syntax-rules ()
    ((my-or) nil)
    ((my-or arg1) arg1)
    ((my-or arg1 arg2) (let ((temp arg1))
                         (if temp temp arg2)))
    ((my-or arg1 arg2 arg3 ***) (my-or (my-or arg1 arg2) arg3 ***))))
}

\subsection{Hygiene}

There is one problem with the definition of \q{my-or} above.
If you have a global variable \q{temp} bound to \q{t},
then the following expression

\q{
(my-or nil temp)
}

does not give the expected result (\q{t}).  This is because

\q{
(my-or nil temp)
}

expands to

\q{
(let ((temp nil))
  (if temp temp temp))
}

which evaluates to \q{nil}.  The temporary lexical variable
\q{temp} introduced in the macro expansion shadows (or {\it
captures\/}) the
global \q{temp} and therefore causes an erroneous result.

The Scheme report requires that the macro definers be {\it
hygienic}, i.e., that they automatically avoid these lexical
captures.  This Common Lisp implementation does {\it not\/}
provide hygiene.  There are two ways out:

1. Use unusual names for any lexical variables you introduce
in the expansion pattern, e.g.,

\q{
...
((my-or arg1 arg2) (let ((__#temp#__ arg1))
                     (if __#temp#__ __#temp#__ arg2)))
...
}

and then hope that nobody ever uses \q{__#temp#__}.

2.  A better alternative is to use a generated symbol that
is guaranteed not to clash with anything Lisp or you can
come up with before or after.  To introduce this gensym,
\q{mbe.lsp} provides a \q{with}-wrapper for the expansion
pattern.  E.g.,

\q{
...
((my-or arg1 arg2) (with ((temp (gensym)))
                     (let ((temp arg1))
                       (if temp temp arg2))))
...
}

Here, the \q{temp} identifier used in the expansion pattern
will be a gensym, and not the symbol literally named
\q{temp}.

\section{{\tt let-syntax} and {\tt letrec-syntax}}

\scmkeyword{either}

\q{define-syntax} defines globally visible macros.  For
local macros (cf.\ Common Lisp's \q{macrolet}), use
\q{let-syntax} or \q{letrec-syntax}.  Example:

\q{
(let-syntax ((either (syntax-rules ()
                       ((either x y) (with ((tmp (gensym)))
                                       (let ((tmp x))
                                         (if tmp tmp y)))))))
  (either "this" "that"))
}

The \q{either} macro is local to the \q{let-syntax} body,
and will not be visible outside.

The Scheme report distinguishes between
\q{let-syntax} and \q{letrec-syntax} the same way that
Common Lisp distinguishes between \q{flet} and \q{labels}.
However, this distinction is not well preserved in \p{mbe.lsp}
because these forms are defined using \q{macrolet}, and
\q{macrolet} doesn't have the same scoping style.

\section{References}

\thebibliography{99}
\bibitem{kohlbecker86} Eugene E. Kohlbecker, Jr., Daniel
P. Friedman, Matthias Felleisen, and Bruce Duba, ``Hygienic
Macro Expansion'', in {\it Proc.\ 1986 ACM Conf.\ on Lisp
and Functional Programming}, pp.\ 151--161.

\bibitem{kohlbecker-thesis} Eugene E. Kohlbecker, Jr.,  {\it
Syntactic
Extensions in the Programming Language Lisp}, PhD thesis,
Indiana Univ., 1986.

\bibitem{r5rs}
\urlh{http://www-swiss.ai.mit.edu/ftpdir/scheme-reports/r5rs-html/r5rs_toc.html}{\it
Revised~(5)
Report on the Algorithmic Language Scheme\/} (``R5RS''), eds.\ Richard
Kelsey, William Clinger, and Jonathan Rees, 1998.
\endthebibliography

\bye
